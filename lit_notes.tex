\documentclass[12pt, oneside]{article}   	% use "amsart" instead of "article" for AMSLaTeX format
\usepackage{geometry}                		% See geometry.pdf to learn the layout options. There are lots.
\geometry{letterpaper}                   		% ... or a4paper or a5paper or ... 
%\geometry{landscape}                		% Activate for rotated page geometry
%\usepackage[parfill]{parskip}    		% Activate to begin paragraphs with an empty line rather than an indent
\usepackage{graphicx}				% Use pdf, png, jpg, or eps§ with pdflatex; use eps in DVI mode
								% TeX will automatically convert eps --> pdf in pdflatex		
\usepackage{amssymb}

%SetFonts

%SetFonts


\title{Notes on lit related to grain hill}
\author{GT}
%\date{}							% Activate to display a given date or no date

\begin{document}
\maketitle

\section*{Furbish et al.~(2009) JGR}

Split flux into product of depth, $h$, and ``flux density'' (a depth-averaged velocity, though they argue not really the same as speed) $\bar{\mathbf{q}}$: $\mathbf{q}_s = h \bar{\mathbf{q}}$

``we envision the en masse motion of a soil as arising from the collective quasi-random motions of soil particles and particle clumps, where the overall motion involves a net downward component that is gravitationally driven.''

Scattering and settling

[reminder to look at role of elastic vs inelastic proportions]

Settling depends on porosity (space into which to settle)

``Details of scattering motions, which generally are more complex, have a less critical role, so the analysis is inherently forgiving of any sins of omission or commission in our treatment of these motions.''

``the flux includes both advective and diffusive parts''

Separates motions into scattering and settling.

Formulates advection-dispersion FP. Novelty is activity probability and modes of motion.

$D$ sits INSIDE the derivatives! Basically, for each mode and direction, one gets:
\begin{equation}
q_x = c a M_1 u_1 - \frac{1}{2} \frac{\partial}{\partial x} ( c a M_1 D_1 ) +  c a M_2 u_2 - \frac{1}{2} \frac{\partial}{\partial x} ( c a M_2 D_2 )
\end{equation}
Here $c$ is concentration, $a$ is the fraction that are active, $M_i$ is the fraction traveling by model $i$ (scattering or settling), $u$ is the drift velocity, and $D$ is diffusivity.

Derives a mean free path that depends on local concentration relative to maximum or consolidated concentration.

Nicely points out that you don't get truly diffusive motion parallel to slope: ``Because gradients in the particle concentration c and the activity probability a parallel to x' are zero, no net diffusive scattering flux parallel to x' occurs.'' 

Formulates the settling flux arising from vertical scattering
\begin{equation}
q_x = q_{z2} \cos^2\theta \frac{\partial \zeta}{\partial x}
\end{equation}
``This, without further ado, describes a slope dependence in the transport arising from a balance of scattering and settling fluxes, going beyond heuristic arguments for such a dependence.''

Right away you can begin to see the link with the grain hill model: in grain hill world, $q_{z2} = \delta d$, and integrating over depth yields $\delta^2 d$, right?

So my $d$ is similar to Furbish's $N_a(z)$: same units, but in grain hill $N_a$ isn't simply $d$ but rather depends on the particle's exposure (so, not just $z$). He also notes that the activity probability $a = N_a \tau$ where $\tau$ is the duration of activity.

He ends up with
\begin{equation}
D = k R h \overline{N_a \left( 1 - \frac{c}{c_m}\right)^2} \cos^2 \theta
\end{equation}
$k=k_A k_2$ is a dimensionless coefficient; $k_2$ relates scattering length scale to mean free path; $k_A$ goes into mean free path and is $1/3$ for spheres. $R$ is particle radius (so, $\delta /2$). The concentrations are a bit tricky to imagine; in grain hill, $c$ is near zero above the surface, and $1$ below. Something like:
\begin{equation}
D = K \delta^2 d \cos^2 \theta
\end{equation}
He also points out that the equivalent of my $K$ represents the ``magnitude of settling motions as determined by the availability and size of local pore spaces into which particles can move.'' $d$ or $N_a$ represents the frequency. Thus, $D_* = D/h$ represents the magnitude-frequency product. He calls $N_a$ the ``activity.''

He then posits exponential decay equations for $N_a$ and $c$. For the grain hill model, his decay-depth scales $\alpha$ and $\beta$ would be of order $\delta$.

``With small a or b relative to h, however, D* is not independent of h. This would introduce to the transport formula (32) additional nonlinearity in h.'' I'm not sure I completely understand this. The math looks reasonable but intuitively it does not make sense. I guess it has to do with the fact that $q_x$ also depends on $h$. When $h$ is much bigger than ``active'' thickness, the flux should intuitively become INDEPENDENT of $h$. I think actually that the grain hill $h$ is effectively $\delta$---implying the biophysically active depth, not the full depth. So then $h$ IS comparable to the decay scale. Right, they note: ``The active soil thickness h does not necessarily coincide with the soil thickness as defined in a pedological sense.''

Interestingly, they illustrate the difference in profile form with and without the $\cos^2\theta$ factor in $D$. The slope gets a bit steeper than parabolic with increasing distance from ridge top.

So, the grain hill formulation DOES capture the spirit of their model inasmuch as it describes slope-normal scattering and vertical settling, even though these are not explicitly broken into categories.

``... details of these lofting motions are not critical.''

Ok, here's an important point: they assume scattering motions do NOT have an advective part. Then note that motions on/near surface, such as gophers. This matters for grain hill because we DO (or can) have an advective part, I think...?

Finally, for grain hill, $N_a$ can have a dependence on slope angle because above 30 degrees grains become activated by gravity as well as scattering.


%\subsection{}



\end{document}  